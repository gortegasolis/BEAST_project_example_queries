% Options for packages loaded elsewhere
\PassOptionsToPackage{unicode}{hyperref}
\PassOptionsToPackage{hyphens}{url}
\PassOptionsToPackage{dvipsnames,svgnames,x11names}{xcolor}
%
\documentclass[
  letterpaper,
  DIV=11,
  numbers=noendperiod]{scrartcl}

\usepackage{amsmath,amssymb}
\usepackage{iftex}
\ifPDFTeX
  \usepackage[T1]{fontenc}
  \usepackage[utf8]{inputenc}
  \usepackage{textcomp} % provide euro and other symbols
\else % if luatex or xetex
  \usepackage{unicode-math}
  \defaultfontfeatures{Scale=MatchLowercase}
  \defaultfontfeatures[\rmfamily]{Ligatures=TeX,Scale=1}
\fi
\usepackage{lmodern}
\ifPDFTeX\else  
    % xetex/luatex font selection
\fi
% Use upquote if available, for straight quotes in verbatim environments
\IfFileExists{upquote.sty}{\usepackage{upquote}}{}
\IfFileExists{microtype.sty}{% use microtype if available
  \usepackage[]{microtype}
  \UseMicrotypeSet[protrusion]{basicmath} % disable protrusion for tt fonts
}{}
\makeatletter
\@ifundefined{KOMAClassName}{% if non-KOMA class
  \IfFileExists{parskip.sty}{%
    \usepackage{parskip}
  }{% else
    \setlength{\parindent}{0pt}
    \setlength{\parskip}{6pt plus 2pt minus 1pt}}
}{% if KOMA class
  \KOMAoptions{parskip=half}}
\makeatother
\usepackage{xcolor}
\setlength{\emergencystretch}{3em} % prevent overfull lines
\setcounter{secnumdepth}{5}
% Make \paragraph and \subparagraph free-standing
\makeatletter
\ifx\paragraph\undefined\else
  \let\oldparagraph\paragraph
  \renewcommand{\paragraph}{
    \@ifstar
      \xxxParagraphStar
      \xxxParagraphNoStar
  }
  \newcommand{\xxxParagraphStar}[1]{\oldparagraph*{#1}\mbox{}}
  \newcommand{\xxxParagraphNoStar}[1]{\oldparagraph{#1}\mbox{}}
\fi
\ifx\subparagraph\undefined\else
  \let\oldsubparagraph\subparagraph
  \renewcommand{\subparagraph}{
    \@ifstar
      \xxxSubParagraphStar
      \xxxSubParagraphNoStar
  }
  \newcommand{\xxxSubParagraphStar}[1]{\oldsubparagraph*{#1}\mbox{}}
  \newcommand{\xxxSubParagraphNoStar}[1]{\oldsubparagraph{#1}\mbox{}}
\fi
\makeatother

\usepackage{color}
\usepackage{fancyvrb}
\newcommand{\VerbBar}{|}
\newcommand{\VERB}{\Verb[commandchars=\\\{\}]}
\DefineVerbatimEnvironment{Highlighting}{Verbatim}{commandchars=\\\{\}}
% Add ',fontsize=\small' for more characters per line
\usepackage{framed}
\definecolor{shadecolor}{RGB}{241,243,245}
\newenvironment{Shaded}{\begin{snugshade}}{\end{snugshade}}
\newcommand{\AlertTok}[1]{\textcolor[rgb]{0.68,0.00,0.00}{#1}}
\newcommand{\AnnotationTok}[1]{\textcolor[rgb]{0.37,0.37,0.37}{#1}}
\newcommand{\AttributeTok}[1]{\textcolor[rgb]{0.40,0.45,0.13}{#1}}
\newcommand{\BaseNTok}[1]{\textcolor[rgb]{0.68,0.00,0.00}{#1}}
\newcommand{\BuiltInTok}[1]{\textcolor[rgb]{0.00,0.23,0.31}{#1}}
\newcommand{\CharTok}[1]{\textcolor[rgb]{0.13,0.47,0.30}{#1}}
\newcommand{\CommentTok}[1]{\textcolor[rgb]{0.37,0.37,0.37}{#1}}
\newcommand{\CommentVarTok}[1]{\textcolor[rgb]{0.37,0.37,0.37}{\textit{#1}}}
\newcommand{\ConstantTok}[1]{\textcolor[rgb]{0.56,0.35,0.01}{#1}}
\newcommand{\ControlFlowTok}[1]{\textcolor[rgb]{0.00,0.23,0.31}{\textbf{#1}}}
\newcommand{\DataTypeTok}[1]{\textcolor[rgb]{0.68,0.00,0.00}{#1}}
\newcommand{\DecValTok}[1]{\textcolor[rgb]{0.68,0.00,0.00}{#1}}
\newcommand{\DocumentationTok}[1]{\textcolor[rgb]{0.37,0.37,0.37}{\textit{#1}}}
\newcommand{\ErrorTok}[1]{\textcolor[rgb]{0.68,0.00,0.00}{#1}}
\newcommand{\ExtensionTok}[1]{\textcolor[rgb]{0.00,0.23,0.31}{#1}}
\newcommand{\FloatTok}[1]{\textcolor[rgb]{0.68,0.00,0.00}{#1}}
\newcommand{\FunctionTok}[1]{\textcolor[rgb]{0.28,0.35,0.67}{#1}}
\newcommand{\ImportTok}[1]{\textcolor[rgb]{0.00,0.46,0.62}{#1}}
\newcommand{\InformationTok}[1]{\textcolor[rgb]{0.37,0.37,0.37}{#1}}
\newcommand{\KeywordTok}[1]{\textcolor[rgb]{0.00,0.23,0.31}{\textbf{#1}}}
\newcommand{\NormalTok}[1]{\textcolor[rgb]{0.00,0.23,0.31}{#1}}
\newcommand{\OperatorTok}[1]{\textcolor[rgb]{0.37,0.37,0.37}{#1}}
\newcommand{\OtherTok}[1]{\textcolor[rgb]{0.00,0.23,0.31}{#1}}
\newcommand{\PreprocessorTok}[1]{\textcolor[rgb]{0.68,0.00,0.00}{#1}}
\newcommand{\RegionMarkerTok}[1]{\textcolor[rgb]{0.00,0.23,0.31}{#1}}
\newcommand{\SpecialCharTok}[1]{\textcolor[rgb]{0.37,0.37,0.37}{#1}}
\newcommand{\SpecialStringTok}[1]{\textcolor[rgb]{0.13,0.47,0.30}{#1}}
\newcommand{\StringTok}[1]{\textcolor[rgb]{0.13,0.47,0.30}{#1}}
\newcommand{\VariableTok}[1]{\textcolor[rgb]{0.07,0.07,0.07}{#1}}
\newcommand{\VerbatimStringTok}[1]{\textcolor[rgb]{0.13,0.47,0.30}{#1}}
\newcommand{\WarningTok}[1]{\textcolor[rgb]{0.37,0.37,0.37}{\textit{#1}}}

\providecommand{\tightlist}{%
  \setlength{\itemsep}{0pt}\setlength{\parskip}{0pt}}\usepackage{longtable,booktabs,array}
\usepackage{calc} % for calculating minipage widths
% Correct order of tables after \paragraph or \subparagraph
\usepackage{etoolbox}
\makeatletter
\patchcmd\longtable{\par}{\if@noskipsec\mbox{}\fi\par}{}{}
\makeatother
% Allow footnotes in longtable head/foot
\IfFileExists{footnotehyper.sty}{\usepackage{footnotehyper}}{\usepackage{footnote}}
\makesavenoteenv{longtable}
\usepackage{graphicx}
\makeatletter
\newsavebox\pandoc@box
\newcommand*\pandocbounded[1]{% scales image to fit in text height/width
  \sbox\pandoc@box{#1}%
  \Gscale@div\@tempa{\textheight}{\dimexpr\ht\pandoc@box+\dp\pandoc@box\relax}%
  \Gscale@div\@tempb{\linewidth}{\wd\pandoc@box}%
  \ifdim\@tempb\p@<\@tempa\p@\let\@tempa\@tempb\fi% select the smaller of both
  \ifdim\@tempa\p@<\p@\scalebox{\@tempa}{\usebox\pandoc@box}%
  \else\usebox{\pandoc@box}%
  \fi%
}
% Set default figure placement to htbp
\def\fps@figure{htbp}
\makeatother

\KOMAoption{captions}{tableheading}
\makeatletter
\@ifpackageloaded{caption}{}{\usepackage{caption}}
\AtBeginDocument{%
\ifdefined\contentsname
  \renewcommand*\contentsname{Table of contents}
\else
  \newcommand\contentsname{Table of contents}
\fi
\ifdefined\listfigurename
  \renewcommand*\listfigurename{List of Figures}
\else
  \newcommand\listfigurename{List of Figures}
\fi
\ifdefined\listtablename
  \renewcommand*\listtablename{List of Tables}
\else
  \newcommand\listtablename{List of Tables}
\fi
\ifdefined\figurename
  \renewcommand*\figurename{Figure}
\else
  \newcommand\figurename{Figure}
\fi
\ifdefined\tablename
  \renewcommand*\tablename{Table}
\else
  \newcommand\tablename{Table}
\fi
}
\@ifpackageloaded{float}{}{\usepackage{float}}
\floatstyle{ruled}
\@ifundefined{c@chapter}{\newfloat{codelisting}{h}{lop}}{\newfloat{codelisting}{h}{lop}[chapter]}
\floatname{codelisting}{Listing}
\newcommand*\listoflistings{\listof{codelisting}{List of Listings}}
\makeatother
\makeatletter
\makeatother
\makeatletter
\@ifpackageloaded{caption}{}{\usepackage{caption}}
\@ifpackageloaded{subcaption}{}{\usepackage{subcaption}}
\makeatother

\usepackage{bookmark}

\IfFileExists{xurl.sty}{\usepackage{xurl}}{} % add URL line breaks if available
\urlstyle{same} % disable monospaced font for URLs
\hypersetup{
  pdftitle={Example queries},
  pdfauthor={Gabriel Ortega},
  colorlinks=true,
  linkcolor={blue},
  filecolor={Maroon},
  citecolor={Blue},
  urlcolor={Blue},
  pdfcreator={LaTeX via pandoc}}


\title{Example queries}
\author{Gabriel Ortega}
\date{}

\begin{document}
\maketitle

\renewcommand*\contentsname{Table of contents}
{
\hypersetup{linkcolor=}
\setcounter{tocdepth}{3}
\tableofcontents
}

\subsection{Note}\label{note}

WORK IN PROGRESS. IF YOU NEED ADDITIONAL INFORMATION TO BE ADDED HERE OR
HAVE ANY REQUESTS, PLEASE EMAIL ME AT ortega\_solis@fzp.czu.cz.

\subsection{About the databases}\label{about-the-databases}

The MOBI lab maintains species occurrence and probability records in SQL
databases. The primary databases are
\textbf{\texttt{MOBI\_atlases\_v1}}, which contains stable datasets, and
\textbf{\texttt{MOBI\_atlases\_testing}}, which holds the most
up-to-date but unverified atlas data. Since SQL is a standardized
language, many R packages can interact with these databases. In this
example, I use \textbf{\texttt{dbplyr}} due to the popularity of the
tidyverse ecosystem. However, you can run the same queries with almost
any other software that supports SQL.

\subsection{Libraries}\label{libraries}

\begin{Shaded}
\begin{Highlighting}[]
\CommentTok{\# Load the packages. Install pacman first in case it is not available}
\NormalTok{pacman}\SpecialCharTok{::}\FunctionTok{p\_load}\NormalTok{(}
\NormalTok{  sf, terra, tidyverse, tidyterra, tictoc, RPostgres, dbplyr, askpass}
\NormalTok{)}
\end{Highlighting}
\end{Shaded}

\subsection{Connect to the database}\label{connect-to-the-database}

Before connecting, remember that most ports in our server are closed for
security reasons (IT policies). The workaround is to open an SSH tunnel
using the following command in a terminal (Powershell for Windows
users):

\begin{verbatim}
ssh YOUR-USER@srv-asus-fzp.science.fzp.czu.cz -L 5432:localhost:5432
\end{verbatim}

Don't close the terminal once the tunnel is open. Linux users can open
the ssh tunnel inside a \texttt{screen} session.

Now you can connect to the database with the following code:

\begin{Shaded}
\begin{Highlighting}[]
\CommentTok{\# Connect to the database}
\NormalTok{con }\OtherTok{\textless{}{-}} \FunctionTok{dbConnect}\NormalTok{(}\FunctionTok{Postgres}\NormalTok{(),}
  \AttributeTok{dbname =} \StringTok{"MOBI\_atlases\_testing"}\NormalTok{,}
  \AttributeTok{host =} \StringTok{"localhost"}\NormalTok{,}
  \AttributeTok{port =} \DecValTok{5432}\NormalTok{,}
  \AttributeTok{user =} \StringTok{"YOUR{-}DATABASE{-}USER{-}NAME"}\NormalTok{,}
  \AttributeTok{password =} \FunctionTok{askpass}\NormalTok{(}\StringTok{"Password: "}\NormalTok{)}
\NormalTok{)}
\end{Highlighting}
\end{Shaded}

\subsection{Example queries}\label{example-queries}

\subsubsection{List tables}\label{list-tables}

List the available tables. The ones of interest are Code Books (CB\_)
with control vocabulary about licenses, models, etc\ldots{} and MOBI
(MOBI\_) that holds the proper records:

\begin{Shaded}
\begin{Highlighting}[]
\FunctionTok{dbListTables}\NormalTok{(con) }\SpecialCharTok{\%\textgreater{}\%}
\NormalTok{  purrr}\SpecialCharTok{::}\FunctionTok{keep}\NormalTok{(}\SpecialCharTok{\textasciitilde{}} \FunctionTok{grepl}\NormalTok{(}\StringTok{"\^{}(CB}\SpecialCharTok{\textbackslash{}\textbackslash{}}\StringTok{\_|MOBI}\SpecialCharTok{\textbackslash{}\textbackslash{}}\StringTok{\_[a{-}z]+)[\^{}0{-}9]*$"}\NormalTok{, .))}
\end{Highlighting}
\end{Shaded}

Unfortunately, dbListTables don't show virtual tables (views). There are
two types of views in our databases:

\begin{enumerate}
\def\labelenumi{\arabic{enumi}.}
\item
  \textbf{Normal views} that execute a simple query and show the
  results.
\item
  \textbf{Materialized views} that are complex queries already executed
  whose results are stored in the database cache.
\end{enumerate}

As a user, you will see the views just as tables. You can check the
views in our databases this way:

\begin{Shaded}
\begin{Highlighting}[]
\NormalTok{query }\OtherTok{\textless{}{-}} \StringTok{"}
\StringTok{  SELECT viewname AS table\_name}
\StringTok{  FROM pg\_views}
\StringTok{  WHERE schemaname = \textquotesingle{}public\textquotesingle{} AND viewname LIKE \textquotesingle{}MOBI\%\textquotesingle{}}
\StringTok{  UNION}
\StringTok{  SELECT matviewname AS table\_name}
\StringTok{  FROM pg\_matviews}
\StringTok{  WHERE schemaname = \textquotesingle{}public\textquotesingle{} AND matviewname LIKE \textquotesingle{}MOBI\%\textquotesingle{}}
\StringTok{"}

\CommentTok{\# Execute the combined query}
\FunctionTok{tbl}\NormalTok{(con, }\FunctionTok{sql}\NormalTok{(query)) }\SpecialCharTok{\%\textgreater{}\%}
  \FunctionTok{pull}\NormalTok{(table\_name)}
\end{Highlighting}
\end{Shaded}

\subsubsection{Check the datasets
available}\label{check-the-datasets-available}

Atlas information, such as \textbf{\texttt{datasetID}},
\textbf{\texttt{licenseID}}, and required \textbf{coauthorships}, should
be primarily verified
\href{https://teams.microsoft.com/l/entity/1c256a65-83a6-4b5c-9ccf-78f8afb6f1e8/_djb2_msteams_prefix_3860493077?context=\%7B\%22channelId\%22\%3A\%2219\%3A83f73536d2d1486796ec7d176d35e415\%40thread.tacv2\%22\%7D&tenantId=f26a48e1-fc21-461a-b97f-ac5bd535f341}{here}.
However, this information is also stored in the MOBI\_dataset table to
provide additional context to the records in the database:

\begin{Shaded}
\begin{Highlighting}[]
\FunctionTok{tbl}\NormalTok{(con, }\StringTok{"MOBI\_dataset"}\NormalTok{)}
\end{Highlighting}
\end{Shaded}

You can also query the table \textbf{MOBI\_vw\_tables\_information} to
see the size of the data tables, column names and other helpful
information. Table names ending with a number are subsections of bigger
tables with the same name. The numbers are the corresponding datasetID.

\begin{Shaded}
\begin{Highlighting}[]
\FunctionTok{tbl}\NormalTok{(con, }\StringTok{"MOBI\_vw\_tables\_information"}\NormalTok{)}
\end{Highlighting}
\end{Shaded}

\subsubsection{Importing data}\label{importing-data}

The most important table for you (as a user) is likely to be
\textbf{MOBI\_vw\_scaled\_presence\_records}. You can check the head of
the table before loading it:

\begin{Shaded}
\begin{Highlighting}[]
\FunctionTok{tbl}\NormalTok{(con, }\FunctionTok{sql}\NormalTok{(}\StringTok{\textquotesingle{}SELECT * FROM "MOBI\_vw\_scaled\_presence\_records"}
\StringTok{             LIMIT 5\textquotesingle{}}\NormalTok{))}
\end{Highlighting}
\end{Shaded}

\paragraph{Using SQL}\label{using-sql}

The following code allows you to import a single atlas dataset according
to its datasetID (Birds of Ontario = 18) from
\textbf{MOBI\_vw\_scaled\_presence\_records} and restrict the records to
its original resolution (scalingID = 1):

\begin{Shaded}
\begin{Highlighting}[]
\CommentTok{\# Importing the Ontario birds atlas data. Check the datasetID in MOBI\_dataset (above)}
\FunctionTok{tic}\NormalTok{()}
\NormalTok{data }\OtherTok{\textless{}{-}} \FunctionTok{tbl}\NormalTok{(con, }\FunctionTok{sql}\NormalTok{(}\StringTok{\textquotesingle{}SELECT * FROM "MOBI\_vw\_scaled\_presence\_records"}
\StringTok{                     WHERE "datasetID" = 18}
\StringTok{                     AND "scalingID" = 1\textquotesingle{}}\NormalTok{)) }\SpecialCharTok{\%\textgreater{}\%}
  \FunctionTok{collect}\NormalTok{()}
\FunctionTok{toc}\NormalTok{()}
\end{Highlighting}
\end{Shaded}

\paragraph{Using tidyverse}\label{using-tidyverse}

I prefer to use SQL inside the tbl function because it makes clear that
the SQL instructions are executed on the server side. However, if you
prefer the tidyverse way, the previous query can be executed like this:

\begin{Shaded}
\begin{Highlighting}[]
\CommentTok{\# Importing the Ontario birds atlas data. Check the datasetID in MOBI\_dataset (above)}
\FunctionTok{tic}\NormalTok{()}
\NormalTok{data }\OtherTok{\textless{}{-}} \FunctionTok{tbl}\NormalTok{(con, }\StringTok{"MOBI\_vw\_scaled\_presence\_records"}\NormalTok{) }\SpecialCharTok{\%\textgreater{}\%}
  \FunctionTok{filter}\NormalTok{(datasetID }\SpecialCharTok{==} \DecValTok{18} \SpecialCharTok{\&}\NormalTok{ scalingID }\SpecialCharTok{==} \DecValTok{1}\NormalTok{) }\SpecialCharTok{\%\textgreater{}\%}
  \FunctionTok{collect}\NormalTok{()}
\FunctionTok{toc}\NormalTok{()}
\end{Highlighting}
\end{Shaded}

\paragraph{Selecting columns}\label{selecting-columns}

If you want to select just a few columns from a table, it is possible to
do it like this (you can infer the tidyverse alternative):

\begin{Shaded}
\begin{Highlighting}[]
\FunctionTok{tic}\NormalTok{()}
\NormalTok{data }\OtherTok{\textless{}{-}} \FunctionTok{tbl}\NormalTok{(con, }\FunctionTok{sql}\NormalTok{(}\StringTok{\textquotesingle{}SELECT}
\StringTok{                     "datasetID",}
\StringTok{                     "scalingID",}
\StringTok{                     "siteID",}
\StringTok{                     "verbatimIdentification",}
\StringTok{                     "startYear",}
\StringTok{                     "endYear" {-}{-}Do not add comma here}
\StringTok{                     FROM "MOBI\_vw\_scaled\_presence\_records"}
\StringTok{                      WHERE "datasetID" = 18}
\StringTok{                      AND "scalingID" = 1\textquotesingle{}}\NormalTok{)) }\SpecialCharTok{\%\textgreater{}\%}
  \FunctionTok{collect}\NormalTok{()}
\FunctionTok{toc}\NormalTok{()}
\end{Highlighting}
\end{Shaded}

\paragraph{Import spatial data}\label{import-spatial-data}

Use sf or terra to import the corresponding grid from the table
\textbf{MOBI\_geometry}.

\begin{Shaded}
\begin{Highlighting}[]
\FunctionTok{tic}\NormalTok{()}
\NormalTok{grid }\OtherTok{\textless{}{-}} \FunctionTok{st\_read}\NormalTok{(con, }\AttributeTok{query =} \StringTok{\textquotesingle{}SELECT * FROM "MOBI\_geometry"}
\StringTok{                WHERE "datasetID" = 18}
\StringTok{                AND "scalingID" = 1\textquotesingle{}}\NormalTok{)}
\FunctionTok{toc}\NormalTok{()}
\end{Highlighting}
\end{Shaded}

\subsection{Session info}\label{session-info}

\begin{Shaded}
\begin{Highlighting}[]
\FunctionTok{sessionInfo}\NormalTok{()}
\end{Highlighting}
\end{Shaded}

\begin{verbatim}
R version 4.4.3 (2025-02-28)
Platform: x86_64-pc-linux-gnu
Running under: Ubuntu 24.04.2 LTS

Matrix products: default
BLAS:   /usr/lib/x86_64-linux-gnu/blas/libblas.so.3.12.0 
LAPACK: /usr/lib/x86_64-linux-gnu/lapack/liblapack.so.3.12.0

locale:
 [1] LC_CTYPE=C.UTF-8     LC_NUMERIC=C         LC_TIME=C           
 [4] LC_COLLATE=C         LC_MONETARY=C        LC_MESSAGES=C       
 [7] LC_PAPER=cs_CZ.UTF-8 LC_NAME=C            LC_ADDRESS=C        
[10] LC_TELEPHONE=C       LC_MEASUREMENT=C     LC_IDENTIFICATION=C 

time zone: Europe/Prague
tzcode source: system (glibc)

attached base packages:
[1] stats     graphics  grDevices datasets  utils     methods   base     

loaded via a namespace (and not attached):
 [1] compiler_4.4.3    fastmap_1.2.0     cli_3.6.4         tools_4.4.3      
 [5] htmltools_0.5.8.1 rstudioapi_0.15.0 yaml_2.3.10       bspm_0.5.7       
 [9] rmarkdown_2.29    knitr_1.50        jsonlite_2.0.0    xfun_0.51        
[13] digest_0.6.37     rlang_1.1.5       evaluate_1.0.3   
\end{verbatim}




\end{document}
